%% Beginning of file 'sample631.tex'
%%
%% Modified 2021 March
%%
%% This is a sample manuscript marked up using the
%% AASTeX v6.31 LaTeX 2e macros.
%%
%%
%% using aastex version 6.3
\documentclass[linenumbers]{aastex631}

%% The default is a single spaced, 10 point font, single spaced article.
%% There are 5 other style options available via an optional argument. They
%% can be invoked like this:
%%
%% \documentclass[arguments]{aastex631}
%% 
%% where the layout options are:
%%
%%  twocolumn   : two text columns, 10 point font, single spaced article.
%%                This is the most compact and represent the final published
%%                derived PDF copy of the accepted manuscript from the publisher
%%  manuscript  : one text column, 12 point font, double spaced article.
%%  preprint    : one text column, 12 point font, single spaced article.  
%%  preprint2   : two text columns, 12 point font, single spaced article.
%%  modern      : a stylish, single text column, 12 point font, article with
%% 		  wider left and right margins. This uses the Daniel
%% 		  Foreman-Mackey and David Hogg design.
%%  RNAAS       : Supresses an abstract. Originally for RNAAS manuscripts 
%%                but now that abstracts are required this is obsolete for
%%                AAS Journals. Authors might need it for other reasons. DO NOT
%%                use \begin{abstract} and \end{abstract} with this style.
%%
%% Note that you can submit to the AAS Journals in any of these 6 styles.
%%
%% There are other optional arguments one can invoke to allow other stylistic
%% actions. The available options are:
%%
%%   astrosymb    : Loads Astrosymb font and define \astrocommands. 
%%   tighten      : Makes baselineskip slightly smaller, only works with 
%%                  the twocolumn substyle.
%%   times        : uses times font instead of the default
%%   linenumbers  : turn on lineno package.
%%   trackchanges : required to see the revision mark up and print its output
%%   longauthor   : Do not use the more compressed footnote style (default) for 
%%                  the author/collaboration/affiliations. Instead print all
%%                  affiliation information after each name. Creates a much 
%%                  longer author list but may be desirable for short 
%%                  author papers.
%% twocolappendix : make 2 column appendix.
%%   anonymous    : Do not show the authors, affiliations and acknowledgments 
%%                  for dual anonymous review.
%%
%% these can be used in any combination, e.g.
%%
%% \documentclass[twocolumn,linenumbers,trackchanges]{aastex631}
%%
%% AASTeX v6.* now includes \hyperref support. While we have built in specific
%% defaults into the classfile you can manually override them with the
%% \hypersetup command. For example,
%%
%% \hypersetup{linkcolor=red,citecolor=green,filecolor=cyan,urlcolor=magenta}
%%
%% will change the color of the internal links to red, the links to the
%% bibliography to green, the file links to cyan, and the external links to
%% magenta. Additional information on \hyperref options can be found here:
%% https://www.tug.org/applications/hyperref/manual.html#x1-40003
%%
%% Note that in v6.3 "bookmarks" has been changed to "true" in hyperref
%% to improve the accessibility of the compiled pdf file.
%%
%% If you want to create your own macros, you can do so
%% using \newcommand. Your macros should appear before
%% the \begin{document} command.
%%
\newcommand{\vdag}{(v)^\dagger}
\newcommand\aastex{AAS\TeX}
\newcommand\latex{La\TeX}

%% Reintroduced the \received and \accepted commands from AASTeX v5.2
%\received{March 1, 2021}
%\revised{April 1, 2021}
%\accepted{\today}

%% Command to document which AAS Journal the manuscript was submitted to.
%% Adds "Submitted to " the argument.
%\submitjournal{PSJ}

%% For manuscript that include authors in collaborations, AASTeX v6.31
%% builds on the \collaboration command to allow greater freedom to 
%% keep the traditional author+affiliation information but only show
%% subsets. The \collaboration command now must appear AFTER the group
%% of authors in the collaboration and it takes TWO arguments. The last
%% is still the collaboration identifier. The text given in this
%% argument is what will be shown in the manuscript. The first argument
%% is the number of author above the \collaboration command to show with
%% the collaboration text. If there are authors that are not part of any
%% collaboration the \nocollaboration command is used. This command takes
%% one argument which is also the number of authors above to show. A
%% dashed line is shown to indicate no collaboration. This example manuscript
%% shows how these commands work to display specific set of authors 
%% on the front page.
%%
%% For manuscript without any need to use \collaboration the 
%% \AuthorCollaborationLimit command from v6.2 can still be used to 
%% show a subset of authors.
%
%\AuthorCollaborationLimit=2
%
%% will only show Schwarz & Muench on the front page of the manuscript
%% (assuming the \collaboration and \nocollaboration commands are
%% commented out).
%%
%% Note that all of the author will be shown in the published article.
%% This feature is meant to be used prior to acceptance to make the
%% front end of a long author article more manageable. Please do not use
%% this functionality for manuscripts with less than 20 authors. Conversely,
%% please do use this when the number of authors exceeds 40.
%%
%% Use \allauthors at the manuscript end to show the full author list.
%% This command should only be used with \AuthorCollaborationLimit is used.

%% The following command can be used to set the latex table counters.  It
%% is needed in this document because it uses a mix of latex tabular and
%% AASTeX deluxetables.  In general it should not be needed.
%\setcounter{table}{1}

%%%%%%%%%%%%%%%%%%%%%%%%%%%%%%%%%%%%%%%%%%%%%%%%%%%%%%%%%%%%%%%%%%%%%%%%%%%%%%%%
%%
%% The following section outlines numerous optional output that
%% can be displayed in the front matter or as running meta-data.
%%
%% If you wish, you may supply running head information, although
%% this information may be modified by the editorial offices.
\shorttitle{Double Tilted Rowland Spectrograph}
\shortauthors{G\"unther et al.}
%%
%% You can add a light gray and diagonal water-mark to the first page 
%% with this command:
%% \watermark{text}
%% where "text", e.g. DRAFT, is the text to appear.  If the text is 
%% long you can control the water-mark size with:
%% \setwatermarkfontsize{dimension}
%% where dimension is any recognized LaTeX dimension, e.g. pt, in, etc.
%%
%%%%%%%%%%%%%%%%%%%%%%%%%%%%%%%%%%%%%%%%%%%%%%%%%%%%%%%%%%%%%%%%%%%%%%%%%%%%%%%%
\graphicspath{{./}{figures/}}
%% This is the end of the preamble.  Indicate the beginning of the
%% manuscript itself with \begin{document}.

\begin{document}

\title{Concept of a Double Tilted Rowland Spectrograph for X-rays}


%% The new \altaffiliation can be used to indicate some secondary information
%% such as fellowships. This command produces a non-numeric footnote that is
%% set away from the numeric \affiliation footnotes.  NOTE that if an
%% \altaffiliation command is used it must come BEFORE the \affiliation call,
%% right after the \author command, in order to place the footnotes in
%% the proper location.
%%
%% Use \email to set provide email addresses. Each \email will appear on its
%% own line so you can put multiple email address in one \email call. A new
%% \correspondingauthor command is available in V6.31 to identify the
%% corresponding author of the manuscript. It is the author's responsibility
%% to make sure this name is also in the author list.
%%

\correspondingauthor{Hans Moritz G\"unther}
\email{hgunther@mit.edu}


\author[0000-0003-4243-2840]{Hans Moritz G{\"u}nther}
\affiliation{MIT Kavli Institute for Astrophysics and Space Research, 77 Massachusetts Avenue, Cambridge, MA 02139, USA}


\author{Add your name behind this, will sort later}
\affiliation{}

\begin{abstract}

This example manuscript is intended to serve as a tutorial and template for
authors to use when writing their own AAS Journal articles. The manuscript
includes a history of \aastex\ and documents the new features in the
previous versions as well as the bug fixes in version 6.31. This
manuscript includes many figure and table examples to illustrate these new
features.  Information on features not explicitly mentioned in the article
can be viewed in the manuscript comments or more extensive online
documentation. Authors are welcome replace the text, tables, figures, and
bibliography with their own and submit the resulting manuscript to the AAS
Journals peer review system.  The first lesson in the tutorial is to remind
authors that the AAS Journals, the Astrophysical Journal (ApJ), the
Astrophysical Journal Letters (ApJL), the Astronomical Journal (AJ), and
the Planetary Science Journal (PSJ) all have a 250 word limit for the 
abstract\footnote{Abstracts for Research Notes of the American Astronomical 
Society (RNAAS) are limited to 150 words}.  If you exceed this length the
Editorial office will ask you to shorten it. This abstract has 182 words.

\end{abstract}

%% Keywords should appear after the \end{abstract} command. 
%% The AAS Journals now uses Unified Astronomy Thesaurus concepts:
%% https://astrothesaurus.org
%% You will be asked to selected these concepts during the submission process
%% but this old "keyword" functionality is maintained in case authors want
%% to include these concepts in their preprints.
\keywords{Classical Novae (251) --- Ultraviolet astronomy(1736) --- History of astronomy(1868) --- Interdisciplinary astronomy(804)}

%% From the front matter, we move on to the body of the paper.
%% Sections are demarcated by \section and \subsection, respectively.
%% Observe the use of the LaTeX \label
%% command after the \subsection to give a symbolic KEY to the
%% subsection for cross-referencing in a \ref command.
%% You can use LaTeX's \ref and \label commands to keep track of
%% cross-references to sections, equations, tables, and figures.
%% That way, if you change the order of any elements, LaTeX will
%% automatically renumber them.
%%
%% We recommend that authors also use the natbib \citep
%% and \citet commands to identify citations.  The citations are
%% tied to the reference list via symbolic KEYs. The KEY corresponds
%% to the KEY in the \bibitem in the reference list below. 

\section{Introduction} \label{sec:intro}
X-ray spectroscopy can deliver many important insights into the astrophysics of the hot and energetic universe that are not otherwise observable. In particular, high-resolution spectroscopy (with resolving power $> 500$ or better) can resolve closely-spaced emission lines which offer diagnostics such as density and temperature, reveal the near-edge fine structure in absorption lines, or allow us to detect the precense of faint and narrow absorption lines caused by the warm-hot intergalactic medium in front of bright background quasars to give just a few examples. In the X-ray band, such high resolving powers can be achieved with either microcalorimaters or dispersive elements such as crystals or diffraction gratings. Microcalorimeters offer a fixed absolute energy resolution $\Delta E$, which gives them good resolving power $R=E/\Delta E$ at large energies $E$ and consequently microcalorimeter detectors are planned for the imaging cameras for several of the next X-ray observatories\cite{charm,athena,lynx papers for detectors). However, diffraction gratings have a distinct advantage for softer X-rays below about 2~keV.

The two largest X-ray observatories in history both field diffraction gratings: The Reflection Grating Spectrometer (RGS) on XMM-Newton\cite{RGS} and both the High-Energy Transmission Grating Spectrograph (HETGS) and the Low-Energy Transmission Spectrograph (LETGS) on Chandra\cite{Carnizares2005,LETG}.
**Are each of these called spectrograph or spectrometer?**
Unlike microcalorimeters, diffraction gratings require an additional dispersive element (a transmission or a reflection grating) in the optical path of the instrument and dedicated detectors to capture the dispersed signal. One common layout for a diffraction spectrometer is the Rowlandtorus\cite{Beuermann:78}. In the plan defined bt the optical axis and the dispersion direction, the gratings are located at one side of a circle, the detectors on the other. This circle is commonly called the ``Rowland circle''. In three dimensions, the surface that defines the best grating positions, make up the surface of a torus, thus the name ``Rowland torus'' for this geometry (see Sect.~\ref{sect:onetorus} for a figure).

In this article, we present an optical layout for an instrument that stacks two or four different channels with distinct optical axes in such a way that each channel follows the Rowland torus prescription, yet at the same time all channels are imaged onto a common set of detectors. The general concept could be applied for a large range of focal lengths, grating constants, and other hardware properties. Thus, we discuss advantanges and challenges of this layout without constraint to any specific mission concept. However, we note that Arcus\cite{}, an X-ray spectroscopy mission selected for phase A study in a NASA midex (medium explorer) call in 2018 applied this layout. Specific details of the  Arcus performace, such as predicted effective area and resolving power are shown \citenum{}, but the general concept of a double tilted Rowland spectrograph with its advantages and drawbacks has not been discussed -- a void we attempt to fill with this article.

The layout of this article is as follows: First, we discuss the properties on the optical elements that are most influential for the design of a DTRS, namely mirror properties, gratings, and detectors in Sect.~\ref{sect:elements}. Then, the illustrate the optical layout, modifying the design of a traditional single-channel Rowland torus spectrograph to a DTRS. In section~\ref{sect:discussion}, we discuss advantages and disadvantages of our DTRS concept. We end with a short summart in section \ref{sect:summary}.

\section{Elements of an X-ray grating spectrograph}
\label{sect:elements}

\subsection{Mirror characteristics and sub-aperturing}

\subsection{Dispersive elements}
In prinicle works for any transmission grating, but makes most sense for CAT gratings, so discuss those

trasmission vs diffraction gratings - Casey 2020

Because X-ray can penetrate most material commonly used to manufacture gratings, X-ray instruments are designed to use the gratings either in transmission or, e.g.\ in the case of the RGS, at very small reflection angles close to the angle of total reflection. Thus, transmission and reflection gratings would be located at essentially the same position in the optical path.


\subsection{Detectors}
The dispersed spectrum is spread out far in the dispersion direction. Thus, the ideal detector is much longer in one direction than in the other. Individual detector elements (e.g.\ CCDs) can be tiled, but to avoid chip gaps, it is beneficial to keep the number of elements low. In our DTRS design several spectral traces are positioned in parallel (section~\ref{sect:opticallayout), so the detector has to be position sensitive in 2D to separate individual channels.  Because ir covers a large area than in direct imaging observations, a detector with moderate cost and power requirements per surface area is well-suited for this design. The detector needs to be slightly curved to follow the Rowland torus. If it is made up of smaller individually flat elements, those elements can be arranged tangential to a circle.

The dispersed spectrum contains fewer photons than direct imaging with the same collecting area and spreads this signal out over a large physical region. Thus, the counts rates are lower. Detectors that might by at risk of pile-up in direct observations can still be used for the dispersed signal. On the other hand, the larger regions needed to extract the signal means that background events (internal background and diffuse astrophysical emission) are more important than in direct imaging observations. Similarly, fast timing resolution is typically not needed, because the lower signal means that the detected photons need to be binned up into longer time bins for analysis than a direct imaging observation. Intrinsic energy resolution is required to both suppress the background by filtering out events that are incompatible with wavelengths of the dispersed signal at any particular location and to separate grating orders, where photons of different orders are diffracted to the same position (e.g.\ the sixth order for the O~{\sc vii} forbidden line of the He-like triplet at 2.21~nm and the seventh order of 1.94~mn photons, close to the O~{\sc viii} Lyman$\alpha$ line).

In our DTRS design, the direct images of all channels are positioned on detectors to provide an accurate wavelength calibration, but also because the high-enery part of the spectrum, which is not efficiently dispersed by gratings can provide valuable context for the scientific interpretation of the high-resolution grating spectrum. Thus, the dector type chosen should also be able to handle direct imaging, albeit at a much lower count rate than in a direct imaging instrument without dispersive elements.

**Ask Eric is he knows a paper to cite for comparison of detector types. Lynx?**

\section{Optical Layout}
In this section, we describe the optical layout for a DTRS. We start from the commonly used layout with a single spectrsocopy channel as it is employed in Chandra/LETGS or XMM/RGS. We then show how this design can be modified step-by-step and what considerations are important when chosing the specific parameters for a design. We illustrastrate the concept with ray-trace simulations. Those simulations are performed using a real ray-trace with the marxs code, but the dispersion angle (which is set by the energy of the photons and the grating constant of the grating) is chosen to be larger than feasible for a real instrument. This is done purely to separate the direct light (zero order) and the dispersed light (positive and negative orders) better in the images, so that the concept can be explained more clearly. Static images are embedded in the article, but we strongly encourage the reader explore the interactive 3D models embedded in the electronic version of the publication (might require Acrobat reader).

f, d, sketches

\subsection{One Rowland Torus with central optical axis}
\label{sect:onetorus}
chip gaps chandra, dither

%\begin{figure}
%  \begin{interactive}{js}{file_wth_arll_components.tar/zip}
%  figure call (e.g.\plotone, \includegraphics, etc.)
%  \end{interactive}
%  \caption{Description of the example figure and the interactive portion which includes how to use the interactive functionality.}
%\end{figure}

\subsection{One tilted Rowland Torus}


This torus is tilted Ralfs tild paper
\cite{doi:10.1117/12.856482}

\subsection{Double tilted Rowland Torus}
For R > r signal dips below plane, fish form, two points, symmetry

two opticla axes,
offset chip gaps
spectroscopic vs imganing focus unless I;ve said that above.


\subsection{Channel offsets}

\subsubsection{Channel offsets in the cross-dispersion direction}

\subsubsection{Channel offsets in the dispersion direction}



\section{Discussion}

\subsection{Subaperturing space considerations, and resolving power}
lower mirror requirements, thus chepaer cost for same resolving power, but can use wdge only, stack channels to get square aperture and get back effective area

\subsection{Number of CCDs}
2 camerase, different directions. Need to get zeroth orders anyway (high-energy, wavelength scale), but more CCDs than just one directions (Lynx), make good use of rea-estate

\subsection{Robustness}
relaxes alignment because multiple channels (thus lower cost and technical risk), resilience because independent channels on different CCDS (see XMM loss of RGS chip), check for systemics in calibration because independent channels, chip gaps

wavelength self-calibration (does not need DTRS, but works there, too). is order sorting and several orders ad one energy

\subsection{Other advantages}

Zero order seen, for direct observation (hard X-rays) and calibration

\subsection{Drawbacks}
increased background due ot larger CCD region, less source-free region for background.instrum analysis, coplexity of analysis (numberof spectra!), sprectra coupled (cross-dispersion, read-out streaks for traditional CCDs, wings and energy resolution), does not work for crowded fields, fields with extended emission (galactic center)


\subsection{Similar concepts}

The Far Ultraviolett Spectroscopic Explorer (FUSE) \cite{2000ApJ...538L...1M} faced a design challenge similar to what we discuss here for X-ray instruments. FUSE also has four spectroscopic channels with gratings arranged in a Rowland geometry. Unlike in X-rays, UV gratings can operate in reflection close to normal incidence. The reflection gratings are thus positioned perpendicular to the path of the rays, which means that are almost tangential to the surface of the Rowland torus. Thus, a small number of gratings (just one per channel in the extreme case) is sufficient and, even if the grating itself is flat, it never deviates much from the surface of the Rowland torus. Since UV light has a longer wavelength, it can be dispersed to larger angles increasing the spectral resolving power. In FUSE, detectors just cover the dispersed signal, and not the direct beam (zeroth order) and detectors for the different channels are independent of each other.

Our design of a DTRS places the direct beam for all channels on a detector. This way, the position of the pointing can be constantly monitored and determined from the observed data. If the pointing position drifts and the optical channels are not fully aligned, this can be seen in the science data. In contrast, FUSE has performed observations where the target was not in the field-of-view of all channels, but the amount of the exposure time lost is unclear \cite{2000ApJ...538L...1M}.

\section{Summary}

A specialized layout for a mission with point source spectroscopy goal, not a general observatory design

Arcus (need ot bring is up finally - or have a section on "Application" obove?)


\begin{acknowledgements}
Support for this work was provided in part through NASA grant NNX17AG43G and Smithsonian Astrophysical Observatory (SAO)
contract SV3-73016 to MIT for support of the {\em Chandra} X-Ray Center (CXC),
which is operated by SAO for and on behalf of NASA under contract NAS8-03060.

\end{acknowledgements}

\software{marxs\cite{}, AstroPy \citep{2013A&A...558A..33A,2018AJ....156..123A}, NumPy \citep{van2011numpy,harris2020array}, Matplotlib \citep{Hunter:2007},IPython\cite{IPython}}

\bibliography{bib}{}
\bibliographystyle{aasjournal}


%% Include this line if you are using the \added, \replaced, \deleted
%% commands to see a summary list of all changes at the end of the article.
%\listofchanges

\end{document}
